\documentclass[letter]{article}
\usepackage[cm]{fullpage}
\usepackage{xcolor}
\usepackage{listings}
\usepackage{graphicx}
\usepackage{hyperref}
%package below is for adding images in exact place in code
\usepackage{float}
\usepackage[toc,page]{appendix}

\graphicspath{ {images/} }

%background color for source code snippets and links 
\definecolor{mygray}{gray}{0.95}

%for changing color in links
\hypersetup{
    colorlinks,
    linkcolor={black},
    citecolor={blue!50!black},
    urlcolor={blue!80!black}
}

%source code with highliting
\lstset{basicstyle=\ttfamily,
	backgroundcolor=\color{mygray},
  	commentstyle=\color{blue},
  	%keywordstyle=\color{blue}
	language=bash,
	xleftmargin=\parindent,
	extendedchars=true, 
	showspaces=false,
	showstringspaces=false,
}

%for an src directory of bash scripts
\newcommand*\lstinputpath[1]{\lstset{inputpath=#1}}
\lstinputpath{src}

%prints out helpful info with layout uncommented below
%\usepackage{layout}

\title{Performance Analysis on Intel XL710 using DPDK i40e Poll Mode Driver \\ Tests on ppc64le using DPDK v17.11}
\author{Mick Tarsel \\  LTC Networking Team \\}  

\begin{document}
%\layout

\maketitle

\newpage
\tableofcontents
\newpage

\section{Machine Information}

The purpose of this test plan is to test Open vSwitch (OvS). 
The document provides commands and the expected output.
This test plan aims to provide a pass/failure reference point based on the output of the command.
Output may change over time due to the nature of this project, however certain information should still be retained from output to signify a pass or failure.
Important information from output will be \setlength{\fboxsep}{0pt}\colorbox{yellow}{{\textcolor{red}{highlited}}}. 
Verify that your output has something similar to the highlited parts of the output.

Each section is self-containing such that you may skip around the document.
After setup and execution of tests, the test concludes with a verification section. 
Highlited output missing from verification section would indicate a failure.
If there is a possibility of a failure refer to the Appendix first.
Make sure you remove previous bridges or interfaces before moving to the next test in case there is a naming conflict.

This test plan utilizes network name spaces (ns) to create virtual network endpoints in order to test the features of OvS. 
Each network name space will be an endpoint on our virtual switch. 
The picture below represents the basic network topology this test plan will use. 

\begin{figure}[H]
\caption{Path of ICMP packet from \texttt{namespace1} through \texttt{br-1} to \texttt{namespace2}}
\hbox{\hspace{-0.5cm} \includegraphics{ping-path} }
\end{figure}

Looking at the diagram above we see 3 name spaces. One root name space, and 2 network name spaces. The veth peers for \texttt{namespace1} are \texttt{tap1} and \texttt{p1}. \texttt{tap1} device exists only inside namespace1. Much similar to a VM having a different Ethernet interface than the host machine. \textit{Please note that root namespace and host are used interchanbly throughout the document.} \texttt{p1} exists on the host and in this picture it has been added as a port to the OvS bridge named \texttt{br-1}. Additional information about name spaces may be found in appendix \ref{appendix:ns}.

Commands below are going to be used most often in this test plan
\begin{itemize}
\item \texttt{ovs-vsctl} : Used for configuring the ovs-vswitchd configuration database (known as ovs-db) 
\item \texttt{ovs-ofctl} : A command line tool for monitoring and administering OpenFlow switches 
%The follow are not used yet
%\item \texttt{ovs-dpctl} : Used to administer Open vSwitch datapaths
\item \texttt{ovs-appctl} : A utility that sends commands to and controls Open vSwitch daemons 
\item \texttt{ovsdb-tool} : Open vSwitch database management utility
\item \texttt{netperf} : a network performance benchmark
\item \texttt{tcpdump} : dump traffic on a network
\item \texttt{netcat} : create, concatenate, and redirect sockets. 
%\item \texttt{ovsdb-server} : TODO
%\item \texttt{ip addr} :  show or manipulate routing, devices, policy routing and tunnels
%\item \texttt{ip netns exec} : process network namespace management
%\item \texttt{modprobe} : Add or remove modules from Linux kernel
%\item \texttt{modinfo} : Show information about a Linux kernel module
%\item \texttt{ps} : Report a snapshot of the current processes
\end{itemize}

Any refernce to another section will be a clickable link to that section.

%%%%%%%%%%%%%%%%%%%%%%%%%%%%%%%%%%%%%%%%%%%%%%%%%%%%%
%%%  RX only bandwidth using testpmd
%%%%%%%%%%%%%%%%%%%%%%%%%%%%%%%%%%%%%%%%%%%%%%%%%%%%%

\section{RX Only Bandwidth using testpmd}
%to many 1 line explanations that should not be indented.
{\setlength{\parindent}{0cm}

\subsection{pktgen Machine (126)}

\begin{lstlisting}
# ./pktgen -l 0-3,8-11,16-19,24-27,32-35,40-43,48-51,56-59,64-67,72-75 --socket-mem=4096,4096 -- -T -P -m "[8:40-43/48-51].0"
\end{lstlisting}

Verify the \texttt{openvswitch.service} is running with systemctl

\subsection{testpmd (124)}

\begin{lstlisting}[escapechar=!]
# Q=8; testpmd -l 0,8,16,24,32,40,48,56,64,72 -m 4096 -w 0002:01:00.0 -- --txd=512 --rxd=512 --mbcache=512 --rxq=$Q--txq=$Q --nb-cores=$Q -i -a --rss-ip  --forward-mode=rxonly  --port-topology=chained
\end{lstlisting}

\begin{lstlisting}
testpmd> show port stats all

  ######################## NIC statistics for port 0  ########################
  RX-packets: 16519665   RX-missed: 48236188   RX-bytes:  3885350400
  RX-errors: 0
  RX-nombuf:  0         
  TX-packets: 16         TX-errors: 0          TX-bytes:  0

  Throughput (since last show)
  Rx-pps:     24898418
  Tx-pps:            0
  ############################################################################
testpmd> 
\end{lstlisting}

\subsection{Starting Test}

\begin{lstlisting}[escapechar=!]
Pktgen:/> start 0
\end{lstlisting}

%%%%%%%%%%%%%%%%%%%%%%%%%%%%%%%%%%%%%%%%%%%%%%%%%%%%%
%%% I/O forward across 2 ports
%%%%%%%%%%%%%%%%%%%%%%%%%%%%%%%%%%%%%%%%%%%%%%%%%%%%%
\section{I/O forward bandwidth using testpmd across ports}
%to many 1 line explanations that should not be indented.
{\setlength{\parindent}{0cm}

\subsection{pktgen Machine (126)}

\begin{lstlisting}
# ./pktgen -l 0-3,8-11,16-19,24-27,32-35,40-43,48-51,56-59,64-67,72-75 --socket-mem=4096,4096 -- -T -P -m "[8:40-43/48-51].0,[56-59/64-67:72].1"
\end{lstlisting}

\subsection{testpmd (124)}

\begin{lstlisting}[escapechar=!]
# testpmd -l 0,8,16,24,32,40,48,56,64,72 -m 4096 -w 0002:01:00.0 -w 0002:01:00.1 -- --txd=512 --rxd=512 --mbcache=512 --rxq=$Q --txq=$Q --nb-cores=$Q -i -a --rss-ip –forward-mode=io
\end{lstlisting}

\subsection{Starting Test}

\begin{lstlisting}[escapechar=!]
Pktgen:/> enable 0 range
-- Pktgen 
Pktgen:/> start 0
\end{lstlisting}

\begin{lstlisting}
Port 0: LSC event
Port 1: LSC event
testpmd> show port stats all
testpmd> show port stats all

  ######################## NIC statistics for port 0  ########################
  RX-packets: 15107923   RX-missed: 62402292   RX-bytes:  4650612480
  RX-errors: 0
  RX-nombuf:  0         
  TX-packets: 30         TX-errors: 0          TX-bytes:  1440

  Throughput (since last show)
  Rx-pps:     20292924
  Tx-pps:            5
  ############################################################################

  ######################## NIC statistics for port 1  ########################
  RX-packets: 30         RX-missed: 0          RX-bytes:  1440
  RX-errors: 0
  RX-nombuf:  0         
  TX-packets: 15108497   TX-errors: 0          TX-bytes:  906509264

  Throughput (since last show)
  Rx-pps:            5
  Tx-pps:     20292567
  ############################################################################
testpmd> 

\end{lstlisting}

%%%%%%%%%%%%%%%%%%%%%%%%%%%%%%%%%%%%%%%%%%%%%%%%%%%%%
%%% I/O forward across 1 port
%%%%%%%%%%%%%%%%%%%%%%%%%%%%%%%%%%%%%%%%%%%%%%%%%%%%%
\section{I/O forward bandwidth using testpmd within same port}
%to many 1 line explanations that should not be indented.
{\setlength{\parindent}{0cm}

\subsection{pktgen Machine (126)}

\begin{lstlisting}
# ./pktgen -l 0-3,8-11,16-19,24-27,32-35,40-43,48-51,56-59,64-67,72-75 --socket-mem=4096,4096 -- -T -P -m "[8-11/16-19:40-43/48-51].0"
\end{lstlisting}

\subsection{testpmd (124)}

\begin{lstlisting}[escapechar=!]
# testpmd -l 0,8,16,24,32,40,48,56,64,72 -m 4096 -w 0002:01:00.0 -- --txd=512 --rxd=512 --mbcache=512 --rxq=$Q --txq=$Q --nb-cores=$Q -i -a --rss-ip --forward-mode=io
\end{lstlisting}

\subsection{Starting Test}

\begin{lstlisting}[escapechar=!]
Pktgen:/> enable 0 range
-- Pktgen 
Pktgen:/> start 0
\end{lstlisting}

\begin{lstlisting}
Port 0: LSC event
testpmd> show port stats all

  ######################## NIC statistics for port 0  ########################
  RX-packets: 27704184   RX-missed: 42793993   RX-bytes:  4229888660
  RX-errors: 0
  RX-nombuf:  0         
  TX-packets: 27704037   TX-errors: 0          TX-bytes:  1662241920

  Throughput (since last show)
  Rx-pps:     23247541
  Tx-pps:     23247573
  ############################################################################
testpmd>
\end{lstlisting}

}%for no indents on one-liners

%%%%%%%%%%%%%%%%%%%%%%%%%%%%%%%%%%%%%%%%%%%%%%%%%%%%%
%%% Beginning of Appendix
%%%%%%%%%%%%%%%%%%%%%%%%%%%%%%%%%%%%%%%%%%%%%%%%%%%%%
\newpage
\begin{appendices}
%to many 1 line explanations that should not be indented.
{\setlength{\parindent}{0cm}

\section{Troubleshooting Installation}
\label{appendix:install}
\subsection{OvS Kernel Module}
If kernel module is not loaded, use modprobe command:
\begin{lstlisting}
# modprobe openvswitch
\end{lstlisting}

Verify that the kernel object exists on your system for your currently booted kernel

\begin{lstlisting}
# ls /lib/modules/$(uname -r)/kernel/net/openvswitch
\end{lstlisting}

\subsection{Starting OvS Service}
If you use \texttt{yum install} for RHEL or \texttt{apt} for Ubuntu, OvS should be started automatically. If not follow these commands:

\subsubsection{RHEL}
Check the status of the OvS service and confirm it is not running:

\begin{lstlisting}
# systemctl status openvswitch
openvswitch.service - LSB: Open vSwitch switch
   Loaded: loaded (/etc/rc.d/init.d/openvswitch; bad; vendor preset: disabled)
   Active: inactive (dead) since Wed 2017-03-22 14:42:55 PDT; 29s ago
     Docs: man:systemd-sysv-generator(8)
\end{lstlisting}

Output verifies that it is \texttt{inactive (dead)}

Then start the service and verify there are new processes started:

\begin{lstlisting}
# systemctl start openvswitch
# systemctl status openvswitch
openvswitch.service - LSB: Open vSwitch switch
   Loaded: loaded (/etc/rc.d/init.d/openvswitch; bad; vendor preset: disabled)
   Active: active (running) since Wed 2017-03-22 14:45:00 PDT; 4 days ago
     Docs: man:systemd-sysv-generator(8)
  Process: 25684 ExecStop=/etc/rc.d/init.d/openvswitch stop 
                               (code=exited, status=0/SUCCESS)
  Process: 25725 ExecStart=/etc/rc.d/init.d/openvswitch start 
                               (code=exited, status=0/SUCCESS)
  CGroup: /system.slice/openvswitch.service
          ├─25748 ovsdb-server: monitoring pid 25749 (healthy)
          ├─25749 ovsdb-server /etc/openvswitch/conf.db -vconsole:emer 
                     -vsyslog:err -vfile:info --remote=punix:/...
          ├─25759 ovs-vswitchd: monitoring pid 25760 (healthy)
          └─25760 ovs-vswitchd unix:/var/run/openvswitch/db.sock -vconsole: 
                              -vsyslog:err -vfile:info --mlock...

\end{lstlisting}

\subsubsection{Ubuntu}
Check the status of the OvS service:
\begin{lstlisting}
# systemctl status openvswitch-switch
openvswitch-switch.service - Open vSwitch
   Loaded: loaded (/lib/systemd/system/openvswitch-switch.service; 
				enabled; vendor preset: enabled)
   Active: inactive (dead) since Wed 2017-03-22 18:07:10 EDT; 22s ago
 Main PID: 5079 (code=exited, status=0/SUCCESS)
\end{lstlisting}
Then start the service and verify there are new processes started:
\begin{lstlisting}
# systemctl start openvswitch-switch
# 
\end{lstlisting}

\begin{lstlisting}[escapechar=!]
# ps -aef | grep ovs
root      6528  6482  grep --color=auto ovs
root     25748     1 !\textcolor{red}{\bf{ovsdb-server}}!: monitoring pid 25749 (healthy)
root     25749 25748 !\textcolor{red}{\bf{ovsdb-server}}! /etc/openvswitch/conf.db  
		-vsyslog:err -vfile:info 
		--remote=punix:/var/run/openvswitch/db.sock 
		--private-key=db:Open_vSwitch,SSL,private_key 
		--certificate=db:Open_vSwitch,SSL,certificate 
		--bootstrap-ca-cert=db:Open_vSwitch,SSL,ca_cert --no-chdir 
		--log-file=/var/log/openvswitch/ovsdb-server.log 
		--pidfile=/var/run/openvswitch/ovsdb-server.pid --detach
root     25759     1  !\textcolor{red}{\bf{ovs-vswitchd}}!: monitoring pid 25760 (healthy)
root     25760 25759  !\textcolor{red}{\bf{ovs-vswitchd}}! unix:/var/run/openvswitch/db.sock 
		-vconsole:emer -vsyslog:err -vfile:info --mlockall --no-chdir 
		--log-file=/var/log/openvswitch/ovs-vswitchd.log 
		--pidfile=/var/run/openvswitch/ovs-vswitchd.pid --detach 
\end{lstlisting}

\subsubsection{Starting from Source Build}
First of all, follow instructions from OvS git repo. Load the kernel module, make sure you have a conf.db file, and start the services:

\begin{lstlisting}
# ps -aef | grep ovs
#
# modprobe openvswitch

# ls /usr/local/etc/openvswitch/
conf.db

# ovsdb-server --remote=punix:/usr/local/var/run/openvswitch/db.sock \
    --remote=db:Open_vSwitch,Open_vSwitch,manager_options \
    --private-key=db:Open_vSwitch,SSL,private_key \
    --certificate=db:Open_vSwitch,SSL,certificate \
    --bootstrap-ca-cert=db:Open_vSwitch,SSL,ca_cert \
    --pidfile --detach --log-file

# ovs-vsctl --no-wait init

# ovs-vswitchd --pidfile --detach --log-file

# ps -aef | grep ovs
root      2131     1  0 16:20 ?        00:00:00 ovsdb-server 
--remote=punix:/usr/local/var/run/openvswitch/db.sock
...
root      2134     1  0 16:20 ?        00:00:00 ovs-vswitchd --pidfile 
--detach --log-file

\end{lstlisting}

My output was shortened but it is clear that ovs has been started. Create a bridge to verify.

\section{Name Spaces}
\label{appendix:ns}
A network name space can be thought of like a container (LXC or Docker) or very tiny virtual machine (VM). Network name spaces have the same idea as a container or VM such that resources are isolated, but a network name space only isolates the network interface. Each network namespace has its own interfaces, routing tables, and forwarding tables. Think of a network name space as a virtual machine with only network functionality. For our specific usage, our name spaces will have an IP address and will be connected to our bridge to run tests.

It is also important to note that processes can be dedicated to a network name space.

Name spaces use virtual Ethernet devices, called veth. The veth device works like a tunnel. This tunnel can be thought of like a Ethernet cable which we will use to connect our name spaces to our bridge. Veth works with peers, one peer is in the name space and the other peer is a port on OvS. 

\subsection{Creating Name Spaces}
To create a name space named \texttt{ns1} type:
\begin{lstlisting}
# ip netns add ns1
\end{lstlisting}

To view your name spaces type:
\begin{lstlisting}
# ip netns list
ns1 (id: 0) 
\end{lstlisting}

\subsection{Deleting Name Space}
\begin{lstlisting}
# ip netns delete ns1
#
\end{lstlisting}

\subsection{Creating Veth Interfaces}
To create a veth device type the following:
\begin{lstlisting}
# ip link add tap1 type veth peer name p1
\end{lstlisting}

You will now have 2 new network interfaces on your machine:
\begin{lstlisting} 
ip addr | grep tap
 p1@tap1: <BROADCAST,MULTICAST,M-DOWN> mtu 1500 qdisc noop state DOWN qlen 1000
 tap1@p1: <BROADCAST,MULTICAST,M-DOWN> mtu 1500 qdisc noop state DOWN qlen 1000
\end{lstlisting}
		
\subsection{Attaching Veth Device to Name Space}
Attach namespace dev \texttt{tap1} to \texttt{ns1} such that \texttt{tap1} exists only in \texttt{ns1}
\begin{lstlisting}
# ip link set tap1 netns ns1
\end{lstlisting}
Interface \texttt{tap1} now exists only in the name space called \texttt{ns1}
\begin{lstlisting}
# ip add | grep tap
#
\end{lstlisting}

Remember that a name space is isolated similar to how a VM would be. So the \texttt{tap1} interface no longer exists in the root's name space, it is in \texttt{ns1} now.

\subsection{Executing Commands in a Name Space}
To view network interfaces inside ns1, execute the ip address  command inside the name space like so:
\begin{lstlisting}
# ip netns exec ns1 ip addr
# ip netns exec ns1 ip link set dev tap1 up
# ip netns exec ns1 ip addr
\end{lstlisting}

Now we have a name space with a network interface. The interface in the name space (\texttt{tap1}) is connected to the interface named \texttt{p1} on the host network. 

\section{Setting Up Name Spaces for Test Environment}
\label{sec:OVSandNS}

To create a name space named \texttt{ns1}:
\begin{lstlisting}
# ip netns add ns1
# ip netns list
ns1
\end{lstlisting}

Create the veth interfaces
\begin{lstlisting}
# ip link add tap1 type veth peer name p1

# ip addr | grep tap
p1@tap1: <BROADCAST,MULTICAST,M-DOWN> mtu 1500 qdisc noop state DOWN qlen 1000
tap1@p1: <BROADCAST,MULTICAST,M-DOWN> mtu 1500 qdisc noop state DOWN qlen 1000
\end{lstlisting}

Move the \texttt{tap1} interface into \texttt{ns1}.

\begin{lstlisting}
# ip netns exec ns1  ip add
1: lo: <LOOPBACK> mtu 65536 qdisc noop state DOWN qlen 1
    link/loopback 00:00:00:00:00:00 brd 00:00:00:00:00:00
# ip link set tap1 netns ns1
# ip netns exec ns1 ip add
1: lo: <LOOPBACK> mtu 65536 qdisc noop state DOWN qlen 1
    link/loopback 00:00:00:00:00:00 brd 00:00:00:00:00:00
381: tap1@if380: <BROADCAST,MULTICAST> mtu 1500 qdisc noop state DOWN qlen 1000
    link/ether ba:15:20:dc:63:ec brd ff:ff:ff:ff:ff:ff link-netnsid 0
# ip addr | grep tap
#
\end{lstlisting}

Set the \texttt{tap1} interface to UP state inside \texttt{ns1} by executing the \texttt{ip addr}. This command is executed inside the name space.

\begin{lstlisting}
# ip netns exec ns1 ip link set dev tap1 up
\end{lstlisting}

Likewise, we can run the same command on the root namepace and turn on the \texttt{p1} interface
\begin{lstlisting}
# ip link set dev p1 up
\end{lstlisting}

Now we have an interface that is up and a virual environment to execute commands in.

\begin{lstlisting}
# ip add | grep p1
 p1@if381: <BROADCAST,MULTICAST,UP,LOWER_UP> mtu 1500 qdisc noqueue state UP 
	qlen 1000
    link/ether 72:9b:17:a0:cb:13 brd ff:ff:ff:ff:ff:ff link-netnsid 0
    inet6 fe80::709b:17ff:fea0:cb13/64 scope link 
       valid_lft forever preferred_lft forever
# ip netns exec ns1 ip add
1: lo: <LOOPBACK> mtu 65536 qdisc noop state DOWN qlen 1
    link/loopback 00:00:00:00:00:00 brd 00:00:00:00:00:00
tap1@if380: <BROADCAST,MULTICAST,UP,LOWER_UP> mtu 1500 qdisc noqueue state UP 
	qlen 1000
    link/ether ba:15:20:dc:63:ec brd ff:ff:ff:ff:ff:ff link-netnsid 0
    inet6 fe80::b815:20ff:fedc:63ec/64 scope link 
       valid_lft forever preferred_lft forever
\end{lstlisting}

\subsection{Troubleshooting: Adding Port to Bridge}

First make sure you have a bridge:
\begin{lstlisting}
# ovs-vsctl show
d35f9ced-06b7-4651-8d1c-f8dd75ff3fc2
    Bridge "br-1"
        Port "br-1"
            Interface "br-1"
                type: internal
    ovs_version: "2.7.0"
\end{lstlisting}

Verify the flows for \texttt{br-1}
\begin{lstlisting}
# ovs-ofctl show br-1
OFPT_FEATURES_REPLY (xid=0x2): dpid:0000c24e5f3e2f4b
n_tables:254, n_buffers:0
capabilities: FLOW_STATS TABLE_STATS PORT_STATS QUEUE_STATS ARP_MATCH_IP
actions: output enqueue set_vlan_vid set_vlan_pcp strip_vlan mod_dl_src 
mod_dl_dst mod_nw_src mod_nw_dst mod_nw_tos mod_tp_src mod_tp_dst
 LOCAL(br-1): addr:c2:4e:5f:3e:2f:4b
     config:     PORT_DOWN
     state:      LINK_DOWN
     speed: 0 Mbps now, 0 Mbps max

\end{lstlisting}

\begin{lstlisting}
#  ovs-ofctl dump-flows br-1
NXST_FLOW reply (xid=0x4):
 cookie=0x0, duration=233.241s, table=0, n_packets=0, n_bytes=0, idle_age=233, 
 priority=0 actions=NORMAL

\end{lstlisting}

Now lets add some ports to our bridge

\begin{lstlisting}
# ovs-vsctl add-port br-1 p1
ovs-vsctl: Error detected while setting up 'p1': 
could not open network device p1 (No such device). 
 See ovs-vswitchd log for details.
ovs-vsctl: The default log directory is "/var/log/openvswitch".
\end{lstlisting}

We added interface \texttt{p1} to bridge \texttt{br-1} however \texttt{p1} interface has not been created on our machine yet.

Unlike a new bridge interface, OvS will not create the network interface for a port. Hoever, OvS will still create the port on the bridge.
\begin{lstlisting}
# ip addr | grep p1
#
\end{lstlisting}

Take a look at your bridge and flows:

\begin{lstlisting}
# ovs-vsctl show
d35f9ced-06b7-4651-8d1c-f8dd75ff3fc2
    Bridge "br-1"
        Port "br-1"
            Interface "br-1"
                type: internal
        Port "p1"
            Interface "p1"
                error: "could not open network device p1 (No such device)"
    ovs_version: "2.7.0"
\end{lstlisting}

The output above shows a similar error after we added the port to the bridge, no such device.

\begin{lstlisting}
# ovs-ofctl show br-1
OFPT_FEATURES_REPLY (xid=0x2): dpid:0000c24e5f3e2f4b
n_tables:254, n_buffers:0
capabilities: FLOW_STATS TABLE_STATS PORT_STATS QUEUE_STATS ARP_MATCH_IP
actions: output enqueue set_vlan_vid set_vlan_pcp strip_vlan mod_dl_src 
mod_dl_dst mod_nw_src mod_nw_dst mod_nw_tos mod_tp_src mod_tp_dst
 LOCAL(br-1): addr:c2:4e:5f:3e:2f:4b
     config:     PORT_DOWN
     state:      LINK_DOWN
     speed: 0 Mbps now, 0 Mbps max
OFPT_GET_CONFIG_REPLY (xid=0x4): frags=normal miss_send_len=0
\end{lstlisting}

We cannot create a flow for an interface that does not exist. The output above verfies no flows have been added for interface p1. Output from \texttt{ovs-ofctl dump-flows br-1} will remain the same as above.

\section{Installing Netperf}
\label{appendix:Netperf}

\begin{lstlisting}
# wget ftp://ftp.netperf.org/netperf/netperf-2.7.0.tar.gz  
# tar -xzf netperf-2.7.0.tar.gz
# cd netperf-2.7.0
# ./configure 
# make 
# make install
\end{lstlisting}

\subsection{Troubleshooting Installation}

If build is not working, you may need to download 2 patches to determine your system arch.

\begin{lstlisting}
# wget -O config.guess \
'http://git.savannah.gnu.org/gitweb/?p=config.git;a=blob_plain;f=config.guess;hb=HEAD' 

# wget -O config.sub \
'http://git.savannah.gnu.org/gitweb/?p=config.git;a=blob_plain;f=config.sub;hb=HEAD' 
\end{lstlisting}

\section{Installing netcat}
\label{appendix:nc}

\begin{lstlisting}
# yum search all netcat
Loaded plugins: langpacks, product-id, search-disabled-repos, subscription-manager
======================= Matched: netcat =============================
nmap.ppc64le : Network exploration tool and security scanner
nmap-ncat.ppc64le : Nmap's Netcat replacement
socat.ppc64le : Bidirectional data relay between two data channels ('netcat++')

# yum install nc
Resolving Dependencies
--> Running transaction check
---> Package nmap-ncat.ppc64le 2:6.40-7.el7 will be installed
--> Finished Dependency Resolution

Dependencies Resolved

Installing:
 nmap-ncat

Installed:
  nmap-ncat.ppc64le 2:6.40-7.el7                                                                                                                                                                                                             

Complete!

\end{lstlisting}


\section{Installing tcpdump}
\label{appendix:tcpdump}

\begin{lstlisting}
# yum search tcpdump
Loaded plugins: langpacks, product-id, search-disabled-repos, subscription-manager
===================== N/S matched: tcpdump =====================
tcpdump.ppc64le : A network traffic monitoring tool

# yum install tcpdump
\end{lstlisting}


}%for no indents on one-liners

\section{Multiple Host Trouble Shooting}
\label{appendix:multihost}

Before you even install Open vSwitch, make sure your 2 machines (physical or virtual) can communicate with each other.

It is recommended to first draw a simple diagram of what your network will look like once configured properly. 
It can be difficult to think of how your network will look given command outputs and different name spaces.

It is important to approach a network problem in pieces or chunks. 
If you cannot communicate with your VM start from the VM's point-of-view and start pinging smaller portions of the network. 
Additionally there may be certain IP Tables or rules on a external switch which may be preventing internet access.
First verify the problem is NOT on the host machine, then you could explore the possibility of an additional rule or firewall somewhere else. 
Don't assume both sides of a connection are working in a network – test the problem from all angles.

Use tools like ping and tcpdump to verify connections. If you are pinging Host B from Host A, start tcpdump listening on Host B to see if the ICMP packet even gets to Host B. This will narrow down the issue in the network.

One very helpful way to see the traffic as it passes through OvS is to use the \texttt{watch} command. 

Here is an example:
\begin{lstlisting}
watch -n.5 "ovs-ofctl dump-flows br-1"
\end{lstlisting}


\subsection{GRE Tunnels}
\label{appendix:gre}

\begin{lstlisting}
# lsmod | grep gre
vport_gre               1889  1 
ip_gre                 23515  1 vport_gre
ip_tunnel              31010  1 ip_gre
gre                     4661  1 ip_gre
openvswitch           149665  4 vport_gre
\end{lstlisting}

To remove all GRE interfaces you will have to unload the ip\_gre module.

\begin{lstlisting}
# rmmod vport_gre
# rmmod ip_gre
# ip adr | grep gre
#
\end{lstlisting}

If vport\_gre module is loaded, remove that as well. This module should be loaded from OVS.

\begin{lstlisting}
# rmmod ip_gre
rmmod: ERROR: Module ip_gre is in use by: vport_gre
# rmmod vport_gre
# rmmod ip_gre

# ip add | grep gre
# 
\end{lstlisting}

\subsection{VXLAN}
\label{appendix:vxlan}

OvS will load its own vxlan kernel module, similar to GRE.

\begin{lstlisting}
# lsmod | grep vxl
vport_vxlan             3631  1
vxlan                  50947  1 vport_vxlan
ip6_udp_tunnel          3283  1 vxlan
udp_tunnel              6023  1 vxlan
openvswitch           137923  4 vport_vxlan
\end{lstlisting}

Make sure the \texttt{vxlan} and the \texttt{vport\_vxlan} kernel module is loaded correctly.

\subsubsection{Flows for VXLAN}
\label{appendix:vxlan-flows}

Flow rules are used to assign VNI to packets travelling into or out of a specific port. 

OpenFlow works similar to iptables such that there are rules inside of a table.

Lets take a simple rule as an example.
\begin{lstlisting}
0: table=0,in_port=11, actions=set_field:100->tun_id,resubmit(,1)
1: table=0,in_port=12, actions=set_field:200->tun_id,resubmit(,1)
2: table=0,actions=resubmit(,1)
\end{lstlisting}
Line 0 defines a new rule in table 0. For any traffic travelling into OpenFlow port 11, set the tun\_id a.k.a the VNI, to 100. Then resubmit the packet to table 1 for further processing. 

So if ns1 was connected to port number 11 one would say, "any traffic travelling from ns1 should be tagged with VNI 100".

Line 1 does something similar, but packets travelling into OpenFlow port 12 will be tagged with the 200 VNI.

Line 2 is a defualt action to send all packets to Table 1. This means it would be a good idea to set a default action in Table 1 to drop all packets that we will not be manipulating.

Other important OpenFlows rules are mentioned below:

\begin{itemize}
\item \texttt{dl\_dst} : The destination MAC address
\item \texttt{nw\_dst} : Destination IP address
\item \texttt{arp} : Handle ARP packets
\end{itemize}

Here is another example of some OpenFlow rules:
\begin{lstlisting}
0: table=1,tun_id=100,arp,nw_dst=10.0.0.1,actions=output:1
1: table=1,tun_id=200,arp,nw_dst=10.0.0.1,actions=output:2

2: table=1,tun_id=100,arp,nw_dst=10.0.0.2,actions=output:10
3: table=1,tun_id=200,arp,nw_dst=10.0.0.2,actions=output:10
4: table=1,priority=100,actions=drop	
\end{lstlisting}

These rules are directing OvS where to send ARP requests. 
Line 0 states any traffic for 10.0.0.1 with VNI 100 will be sent to OpenFlow port 1. 
Line 1 is similar but will send packets with VNI 200 to port 2.

Line 2 says that an ARP packet with a VNI of 100 and destination address of 10.0.0.2 will be sent to OpenFlow port 10.

Line 4 is a default drop rules.

\end{appendices}

\end{document}

